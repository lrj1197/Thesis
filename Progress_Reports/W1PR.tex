\documentclass[12pt]{article}

\begin{document}

\noindent Lucas Jameson\\
Jan 11, 2019 \\
Week 1 Progress Report \\

Summary - This week was spent designing the cooling apparatus for the solidification of ammonia (NH3). Two types of apparatuses were thought of: a two phase cooling helix that cools a bath of $86\%$ methanol (MeOH) to a temperature of $-80$ C which is below the freezing point (FP) of NH3. The other is a cold-finger that is submerged in liquid Nitrogen (LN2) and is in direct contact with the cryostat.

Results - The cold-finger method was tested first due to the fact that all the materials that were used to build it were in the lab. The first prototype used a $3$ in long Aluminum (Al) finger attached to a $0.25x1.25$in Al plate that was mounted to the bottom of the cryostats end cap. The method for fastening was to mill out two holes on the plate that the screws that hold the end cap to the cryostat and grip onto. The cold-finger was attached to the mounting plate via araldite epoxy. To test this prototype a $8x6x12$in styrofoam box was filled with approximately $1$ L of LN2. The cold-finger was submerged in the LN2 and kept at an elevation such that half the cold-finger was in the LN2. The top of the box was capped with a foam block. AB6 resistor was used to measure the temperature half way up the exterior of the cryostat. The lowest temperature that was achieved using this method was $~ -70$ C. A Computer Animated Drawing (CAD) of the cold-finger and mounting plate is on Github under the $lrj1197$ profile and in the sub-repository CAD and titled coldfinger_1_2.sat.

Further Goals - The cold-finger method was unable to get the cryostat cold enough for solidifying NH3. The temperature of interior walls of the cryostat were not measured and getting a measurement would be good to know. Since Al has a lower thermal conductivity than that of Copper which was used in Temperature-controlled crystal formation with a new prototype cryogenic device: experimental set up and simulations paper, prototype 2 will be built to have a larger mounting plate for better thermal conduction from cryostat to LN2 bath, and more cold-fingers will be added. 

\end{document}