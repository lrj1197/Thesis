\documentclass[12pt]{article}

\begin{document}

\noindent Lucas Jameson\\
Jan 18, 2019 \\
Week 2 Progress Report \\

Summary - Based on the work of last week, a cold finger redesign was constructed and tested. A third redesign is required to obtain better cooling. Considering the dewar for the cooling process, several modifications are needed to make this more efficient.  

Results - Prototype II was successful in cooling down the bottom of the cryostat to temperatures under $-77$ C. The temperature reached was $-82$ C and held the temperature for around one hour. Based on the success of prototype II the cold-finger method will be adequate for slow freezing NH$3$. In order to minimize the time spent cooling down the system another prototype is needed. Prototype III will be made of copper, and instead of having $4$ small fingers, one $1x1x3$in finger will be mounted. Copper will be used because it has a larger thermal conductivity than Aluminum which means its better at conducting heat. The thermal conductivity of Copper is $386$ W/mK and of Aluminum is $204$ W/mK. The $4$L dewar seems to work well but just need a better cover. The cover needs a port for pouring LN$2$ in, measuring the depth of LN$2$, and is modular for easy access to the cryostat. 

Further Goals - Assemble prototype III, model the thermodynamics of the system to learn what the temperature is inside the cryostat based on the temperature outside, and add level-meter for LN$2$ in dewar. 

\end{document}


